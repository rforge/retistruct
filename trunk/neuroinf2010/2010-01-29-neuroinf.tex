\documentclass[final,hyperref={pdfpagelabels=false}]{beamer} 
\mode<presentation> {  %% check http://www-i6.informatik.rwth-aachen.de/~dreuw/latexbeamerposter.php for examples
  \usetheme{Sterratt}    %% you should define your own theme e.g. for big headlines using your own logos 
}
% \usepackage[garamond]{mathdesign}
\usepackage[english]{babel}
\usepackage[latin1]{inputenc}
\usepackage[T1]{fontenc}
% \usepackage{amsmath,amsthm, amssymb, latexsym}
\usepackage{helvet}%\usefonttheme{professionalfonts}  % times is
% obsolete
\usepackage[garamond]{mathdesign}
\renewcommand{\rmdefault}{ugm}
\renewcommand{\sfdefault}{ugm}
\newcommand{\figack}[1]{{\par\small\vskip -0.5ex\hfill{\color{blue} #1}\par}}
\usepackage{natbib}
\makeatletter
\def\newblock{\beamer@newblock}
\makeatother 


% \usefonttheme[onlymath]{serif}
\boldmath
\usepackage[orientation=landscape,size=a4,debug]{beamerposter}                       % e.g. for DIN-A0 poster
% \usepackage[orientation=portrait,size=a1,scale=1.4,grid,debug]{beamerposter}                  % e.g. for DIN-A1 poster, with optional grid and debug output
% \usepackage[size=custom,width=200,height=120,scale=2,debug]{beamerposter}                     % e.g. for custom size poster
% \usepackage[orientation=portrait,size=a0,scale=1.0,printer=rwth-glossy-uv.df]{beamerposter}   % e.g. for DIN-A0 poster with rwth-glossy-uv printer check
% ...
% 
\title{Inference of original retinal coordinates from flattened
  retinae} \author{David C Sterratt$^1$ and Ian D Thompson$^2$}
\institute{$^1$Institute for Adaptive \& Neural Computation, School of
  Informatics, University of Edinburgh\\ $^2$MRC Centre for Developmental
  Neurobiology, King's College London} \date{Jul. 31th, 2007}
\begin{document}

\begin{frame}{} 
  \begin{columns}[T]

    %%%%%%%%%%%%%%%%%%%%%%%%%%%%%%%%%%%%%%%%%%%%%%%%%%
    \begin{column}{.33\linewidth}

      \begin{block}{Abstract}
        In retrograde tracing experiments to determine the mapping of
        connections from the retina to the superior colliculus in
        mammals (see e.g. \citealp{RashEtal05oppo}), fluorescent
        tracer is injected at a point in the superior colliculus and
        allowed to transport retrogradely down the axons of retinal
        ganglion cells to their cell bodies in the retina. The retina
        is then dissected and flattened, and the pattern of dye in
        cell bodies can be seen in the flattened retina. In the
        process of flattening the retina, a number incisions are made
        and tearing occurs. The pattern of label can be spread across
        incisions and tears in the flattened retina, complicating
        analysis of the mapping.

        One way of simplifying the analysis would be to infer the
        positions of the cell bodies in the spherical coordinate
        system of the intact retina from their positions in the
        flattened retina. We present a method to achieve this
        approximately by minimisation of an energy function. A
        triangular grid is laid over the flattened retina. The
        coordinates of the grid are then transformed so that they lie
        on a partial sphere with the correct dimensions for the intact
        retina, and the transformation is then adjusted so as to
        minimise the sum of the squared differences between the
        lengths between corresponding pairs of adjacent points on the
        flattened and intact retinae. This also allows for lines of
        latitude and longitude in the intact retina to be projected
        onto the flattened retina.
      \end{block}
      

      \begin{block}{The problem}

        Example of data from retrograde labelling studies in the
        developing visual system:
        
        \includegraphics[width=0.49\linewidth]{../figures/UptoEtal07emer1}
        \includegraphics[width=0.49\linewidth]{../figures/UptoEtal07emer}
        \figack{\citet{UptoEtal07deve}}

        \begin{itemize}
        \item \textbf{Problem: cell bodies which were neighbours in the original
          retina may be far apart in the flattened retina}
        \item Solution: morph the flattened retina back onto a sphere
        \end{itemize}
      \end{block}

    \end{column}

    %%%%%%%%%%%%%%%%%%%%%%%%%%%%%%%%%%%%%%%%%%%%%%%%%%
    \begin{column}{.33\linewidth}

      \begin{block}{Method: Stitching and triangulation}
        \begin{columns}
          \begin{column}{0.35\linewidth}
            \includegraphics[width=\linewidth]{../figures/M634-4-triangulated-stitched2}            
          \end{column}
          \begin{column}{0.6\linewidth}
            \begin{itemize}
            \item Incisions and tears marked up by expert and stitched
              automatically by pairing points at equivalent fractional
              distances along each incision or tear
            \item Delaunay Triangulation of points on retinal outline and
              randomly-generated internal points
            \item Iterative procedure to detect and remove ``flaps''
            \end{itemize}
          \end{column}
        \end{columns}
      \end{block}


      \begin{block}{Method: Initial projection}
        \begin{columns}
          \begin{column}{0.35\linewidth}
            \includegraphics[width=\linewidth]{../figures/M634-4-initial-proj2}
          \end{column}
          \begin{column}{0.5\linewidth}
            \includegraphics[width=0.85\linewidth]{../figures/M634-4-initial-proj-3d2}  

          \end{column}
        \end{columns}
            \begin{itemize}
            \item Project grid onto sphere curtailed at latitude of 50$^\circ$
              \begin{itemize}
              \item Radius determined by area of the flattened retina.
              \end{itemize}
            \item Fix points on rim of flattened retina to the
              latitude of the rim 
            \item Cell bodies projected using barycentric coordinates
            \end{itemize}

      \end{block}

      \begin{block}{Method: Final projection}
        \begin{columns}
          \begin{column}{0.35\linewidth}
            \includegraphics[width=\linewidth]{../figures/M634-4-final-proj2}
          \end{column}
          \begin{column}{0.5\linewidth}
            \includegraphics[width=0.85\linewidth]{../figures/M634-4-final-proj-3d2}

            % \item The aim now is to infer the latitude $\phi_i$ and longitude
            %   $\lambda_i$ on the sphere to which each grid point $i$ on the
            %   flattened retina is projected.

          \end{column}
        \end{columns}

            \begin{itemize}
            \item Infer optimal projection onto sphere by changing
              location of vertices on sphere so as to minimise an
              energy function $E$ with two components:
              \begin{itemize}
              \item Sum of squared differences between lengths $L_i$
                and $l_i$ of corresponding connections $\mathcal{C}$
                in flattened and spherical retina
              \item Sum of exponentials of ratios of areas  of 
                triangles $\mathcal{T}$ in flattened retina $A_i$ and
                on sphere $a_i$
              \item  $E = \frac{1}{2} \sum_{i\in\mathcal{C}} \frac{(l_i -
              L_i)^2}{L_i}  + \alpha\sum_{i\in\mathcal{T}} \exp(-k\frac{a_i}{A_i})
            $
            
              \end{itemize}
            \end{itemize}

      \end{block}

%       \begin{block}{Problems}
%         \begin{itemize}
%         \item Method for folding flattened retina onto hemisphere has
%           been developed 
%         \end{itemize}
%       \end{block}


    \end{column}

    %%%%%%%%%%%%%%%%%%%%%%%%%%%%%%%%%%%%%%%%%%%%%%%%%% 
    \begin{column}{.33\linewidth}

      \begin{block}{Validation}
        \begin{columns}
          \column[T]{0.7\linewidth}

          \includegraphics[width=0.33\linewidth]{../data/14-12-09_1754.jpg}
          \includegraphics[width=0.33\linewidth]{../data/14-12-09_1755.jpg}
          \includegraphics[width=0.33\linewidth]{../data/14-12-09_1758.jpg}

          \includegraphics[width=0.33\linewidth]{../figures/orange-no-grid.png}
          % \includegraphics[width=0.33\linewidth]{../figures/orange-outline.pdf}
          % \includegraphics[width=0.33\linewidth]{../figures/orange-stitched.pdf}
          % \includegraphics[width=0.33\linewidth]{../figures/orange-stitched-triangulated.pdf}
          \includegraphics[width=0.33\linewidth]{../figures/orange-many-grid.pdf}
          \includegraphics[width=0.33\linewidth]{../figures/orange-grid.png}


          \column[T]{0.25\linewidth}
          % \includegraphics[width=\linewidth]{../figures/orange-projected.pdf}
          \includegraphics[width=\linewidth]{../figures/orange-final.pdf}

        \end{columns}
        \begin{itemize}
        \item While the simulated retina does not have the same
          material properties as a real retina, this experiment does
          suggest that the reconstruction method has a reasonable
          degree of accuracy.
        \end{itemize}

      \end{block}

      \begin{block}{Discussion \& Conclusions}
        \begin{itemize}
        \item Method for folding flattened retina onto partial sphere has
          been developed, and validation study suggests it is
          reasonably accurate.
        \item Allows the location of cell bodies to be inferred in
          the original near-spherical coordinates, which is especially
          useful when groups of cells cross incisions.
        \item At present, the method needs supervision to indicate
          where incisions and tears are
        \item Possible unsupervised alternative approach to stitching:
          attraction between edges
          \begin{itemize}
          \item This was tried, but proved complex to implement in a
            way such that the correct mappings were made
          \end{itemize}
        \item Applications
          \begin{itemize}
          \item Statistics of retrograde tracing
          \item Allowing electrode stimulation arrays to map activity
            in more realistic fashion onto flattened retina
          \end{itemize}
        \item Next steps:
          \begin{itemize}
          \item Test on more data
          \item Speed and reliability improvements to the optimisation
            procedure.
          \end{itemize}
        \end{itemize}
      \end{block}

      \begin{block}{Acknowledgements}

        KCL: Daniel Nedergaard, Ian Thompson, Andrew Lowe

        ANC: Michael Herrmann, David Willshaw, Matthias Hennig

        This work was funded by the Wellcome Trust.
      \end{block}


      \begin{block}{References}
        \footnotesize
        \bibliographystyle{apalike}
        \bibliography{mystrings,my}
      \end{block}
    \end{column}


  \end{columns}



\end{frame}


\end{document}

% LocalWords:  colliculus RashEtal oppo UptoEtal deve Delaunay KCL Nedergaard
% LocalWords:  ANC Herrmann Willshaw Hennig Wellcome mystrings
