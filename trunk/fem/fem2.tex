\documentclass{article}

\usepackage{amsmath}

\title{Computing the strain energy of a triangle}
\renewcommand{\vec}[1]{\mathbf{#1}}
\renewcommand{\epsilon}{\varepsilon}

\begin{document}

\section{The problem}
\label{fem:sec:problem}

Suppose we have a triangle in a 2D object whose verticies are at
points
\begin{displaymath}
  \vec{p}_i,\quad
  \vec{p}_j,\quad
  \vec{p}_k
\end{displaymath}
and these are moved to points 
\begin{displaymath}
  \vec{q}_i=\vec{p}_i+\vec{a}_i,\quad
  \vec{q}_j=\vec{p}_j+\vec{a}_j,\quad
  \vec{q}_k=\vec{p}_k+\vec{a}_k
\end{displaymath}
in 3D space.  Assuming that the triangle is isotropic with Young's
modulus $E$ and a Poisson's ratio $\nu$, we wish to compute the strain
energy of the deformed triangle as a function of $\vec{p}_i,
\vec{p}_j, \vec{p}_k, \vec{q}_i, \vec{q}_j$ and $\vec{q}_k$. This will
then be differentiated with respect to $\vec{q}_i, \vec{q}_j$ and
$\vec{q}_k$ in order to determine the force on each node.

\section{The FEM method}
\label{fem2:sec:fem-method}

The properties of the material are described by the elasticity matrix
$\vec{D}$. For plane, isotropic stress, the elasticity matrix is:
\begin{displaymath}
  \vec{D} = \frac{E}{1-\nu^2}
  \begin{pmatrix}
    1 & \nu & 0 \\
    \nu & 1 & 0 \\
    0   & 0 & (1-\nu)/2
  \end{pmatrix}
\end{displaymath}
The Poisson ratio $\nu$ is dimensionless and Young's modulus $E$ has
units of pressure (Force per unit area).

The energy due to a displacment $\vec{a}$ of an element depends on the
material properties and the shape of the element. More quantitatively,
the energy is given by:
\begin{displaymath}
  \frac{1}{2}\vec{a^T}\vec{K}\vec{a}
\end{displaymath}
where $\vec{K}$ is the stiffness matrix (to be defined). The force
corresponding to the displacment vector is
\begin{displaymath}
  \vec{f} = \vec{K}\vec{a}  
\end{displaymath}

The stiffness matrix depends
on the elasticity matrix $\vec{D}$ (defined above) and a matrix
$\vec{B}$ that depends on the shape of the element:
\begin{displaymath}
  \vec{K} = \vec{B^T}\vec{D}\vec{B}
\end{displaymath}
For a triangle with verticies at $(x_1, y_1)$, $(x_2, y_2)$ and $(x_3,
y_3)$ the shape matrix $\vec{B}$ is
\begin{displaymath}
  \vec{B} = \frac{1}{2\Delta}\vec{B_0}
\end{displaymath}
where
\begin{displaymath}
  \vec{B_0} = \frac{1}{2\Delta}
  \begin{pmatrix}
    b_1 & 0  & b_2 & 0 & b_3 & 0 \\
    0  & c_1 & 0  & c_2 & 0 & c_3 \\
    c_1 & b_1 & c_2 & b_2 & c_3 & b_3
  \end{pmatrix}
\end{displaymath}
where
\begin{displaymath}
  2\Delta =
  \begin{vmatrix}
    1 & x_1 & y_1 \\
    1 & x_2 & y_2 \\
    1 & x_3 & y_3 \\
  \end{vmatrix}
\end{displaymath}
where
\begin{displaymath}
  \begin{array}{lll}
    a_1 = x_2y_3 - y_2x_3 &
    b_1 = y_2 - y_3 &
    c_1 = x_3 - x_2 \\
    a_2 = x_3y_1 - y_3x_1 &
    b_2 = y_3 - y_1 &
    c_2 = x_1 - x_3 \\
    a_3 = x_1y_2 - y_1x_2 &
    b_3 = y_1 - y_2 &
    c_3 = x_2 - x_1 
  \end{array}
\end{displaymath}

\section{Change of frame of reference}
\label{fem:sec:workings-so-far}

The transformation of a triangle between the 2D and 3D frames of
reference can be decomposed into a rotation, and stretching and
shearing. The energy function given by the FEM is not
rotation-invariant, so this rotation needs to be factored out. This
section describes how this is done.

Define the matricies:
\begin{displaymath}
  \vec{P_0} = \left(\vec{p}_j-\vec{p}_i, \vec{p}_k-\vec{p}_i\right) \quad
  \vec{Q_0} = \left(\vec{q}_j-\vec{q}_i, \vec{q}_k-\vec{q}_i\right)    
\end{displaymath}
$\vec{P_0}$ is a 2-by-2 matrix, and $\vec{Q_0}$ is a 3-by-2
matrix. The columns of these matricies are the coordinates of two of
the verticies of the triangle, assuming the first vertex is at the
origin.

Determine the 3-by-2 matrix $\vec{M}$ representing the transformation
of the triagle in the 2D frame of reference to the 3D frame of
reference:
\begin{displaymath}
  \vec{Q_0} = \vec{M}\vec{P_0} \quad \mbox{hence} \quad
  \vec{M} = \vec{Q_0}\vec{P_0}^{-1}
\end{displaymath}
Using polar decomposition, factorise $\vec{M}$ into a 3-by-2 unitary
rotation matrix $\vec{R}$ and a 2-by-2 positive semidefinite Hermitian
matrix $\vec{H}$:
\begin{displaymath}
  \vec{M} = \vec{R}\vec{H}
\end{displaymath}

We wish to find the locations of the points in the 3D frame if they
are rotated back into the 2D frame of reference. 
Compute the displacements in the original frame of reference
\begin{displaymath}
  \vec{P'_0} = \vec{R^{-1}}\vec{Q_0} 
\end{displaymath}
The elements of $\vec{P'_0}$ and $\vec{P_0}$ are identified as:
\begin{displaymath}
  \vec{P_0} =  
  \begin{pmatrix} 
    x_2 & x_3 \\
    y_2 & y_3 \\
  \end{pmatrix}
  \quad
  \vec{P'_0} =  
  \begin{pmatrix} 
    x'_2 & x'_3 \\
    y'_2 & y'_3 \\
  \end{pmatrix}
\end{displaymath}



\section{The forward FEM method}
\label{fem2:sec:forward-fem-method}

Using this idendification, the displacement vector 
$\vec{a}$ is constructed as follows:
\begin{displaymath}
  \vec{a} = (0, 0, x'_2-x_2, y'_2-y_2, x'_3-x_3, y'_3-y_3)
\end{displaymath}
The stiffness matrix $\vec{K}$ is derived from the coordinates of the
triangle in the 2D frame: $(0,0)$, $(x_2, y_2)$ and  $(x_3, y_3)$.

Find the forces $\vec{f'}$ in the original frame using:
\begin{displaymath}
  \vec{f'} = \vec{K}\vec{a}
\end{displaymath}
Separate forces back into individual forces, and convert to the 3D
frame of reference:
\begin{displaymath}
  \vec{F} = \vec{R}\vec{F'}
\end{displaymath}

\section{The reverse FEM method}
\label{fem2:sec:reverse-fem-method}

First we need to find the stiffness $\vec{K'}$ of the triangle from
the 3D frame of reference, which has been transformed into the 2D
frame. We use the coordinates $(0,0)$, $(x'_2, y'_2)$ and $(x'_3,
y'_3)$. The displacement vector 
$\vec{a'}$ is constructed as follows:
\begin{displaymath}
  \vec{a'} = (0, 0, x_2-x'_2, y_2-y'_2, x_3-x'_3, y_3-y'_3)
\end{displaymath}
The energy is
\begin{displaymath}
  \vec{a'^T}\vec{K'}\vec{a'} = \frac{1}{(2\Delta')^2}
  \vec{a'^T}\vec{B_0'^T}\vec{D}\vec{B_0'}\vec{a'}
\end{displaymath}
To obtain the force in each direction, this needs to be differentiated
by $x'_1, x'_2\dots$. This can be done by parts.

\subsection{Differentiation by $x_1$}
\label{fem2:sec:differentiation-x_1}

\begin{displaymath}
  2\frac{\partial \Delta'}{\partial x'_1} = y'_2 - y'_3
\end{displaymath}
Hence 
\begin{displaymath}
  \frac{\partial}{\partial x'_1}\left(\frac{1}{(2\Delta')^2}\right) = 
  -2\frac{y'_2 - y'_3}{(2\Delta')^3}
\end{displaymath}
\begin{displaymath}
  \frac{\partial \vec{B'_0}}{\partial x'_1} = 
  \begin{pmatrix}
    0 & 0 & 0 & 0 & 0  & 0 \\
    0 & 0 & 0 & 1 & 0  & -1 \\
    0 & 0 & 1 & 0 & -1 & 0
  \end{pmatrix}
\end{displaymath}

\section{Old stuff}
\label{fem2:sec:old-stuff}

The original coordinates of a point in the triangle and its
displacement are defined as:
\begin{displaymath}
  \vec{p}= \begin{pmatrix}
    x \\
    y
  \end{pmatrix}
\quad\mbox{and}\quad
\vec{u}= \begin{pmatrix} 
  u \\
  v
\end{pmatrix}
\end{displaymath}
respectively.

The strain associated with that point is defined as:
\begin{displaymath}
  \vec{\epsilon} =
  \begin{pmatrix}
    \epsilon_x \\ \epsilon_y \\ \gamma_{xy}
  \end{pmatrix}
  = 
  \begin{pmatrix}
    \frac{\partial u}{\partial x} \\
    \frac{\partial v}{\partial y} \\
    \frac{\partial u}{\partial y} +
    \frac{\partial v}{\partial u} 
  \end{pmatrix}
\end{displaymath}
The strain is dimensionless.

The stress causing that strain is denoted $\vec{\sigma}$, and is
related to the strain by the elasticity matrix D:
\begin{displaymath}
  \vec{\sigma} =
  \begin{pmatrix}
    \sigma_x \\ \sigma_y \\ \tau_{xy}
  \end{pmatrix}
  = 
  \vec{D}\vec{\epsilon}
\end{displaymath}

The strain energy density is
\begin{displaymath}
  \frac{1}{2}\vec{\sigma}^T\vec{\epsilon} = 
  \frac{1}{2}\vec{\epsilon}^T\vec{D}\vec{\epsilon}
\end{displaymath}


\end{document}
