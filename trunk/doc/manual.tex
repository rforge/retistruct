\documentclass{article}

\usepackage[british]{babel}

\title{The retinal mapping data and the nmf\_morph toolkit}

\begin{document}

\maketitle

\section{Data format}
\label{manual:sec:reading-data}

\begin{table}
  \begin{tabular}{ll}
    \texttt{MAPNUM}   & \\  
    \texttt{MINLAT}   & \\
    \texttt{MAXLAT}   & \\
    \texttt{MINLON}   & \\
    \texttt{MAXLON}   & \\
    \texttt{LABLAT}   & \\
    \texttt{LABLON}   & \\
    \texttt{XRED}     & $x$-coordinates of red-labeled cell bodies    \\
    \texttt{YRED}     & $y$-coordinates of red-labeled cell bodies    \\
    \texttt{XGREEN}   & $x$-coordinates of green-labeled cell bodies  \\
    \texttt{YGREEN}   & $y$-coordinates of green-labeled cell bodies  \\
    \texttt{XDOUBLE}  & $x$-coordinates of doubly-labeled cell bodies \\ 
    \texttt{YDOUBLE}  & $y$-coordinates of doubly-labeled cell bodies \\
    \texttt{XGRID}    & \\
    \texttt{YGRID}    & \\
    \texttt{PERIM}    & \\
    \texttt{AREA}     & \\
    \texttt{GRIDX}    & \\
    \texttt{GRIDY}    & \\
    \texttt{XGRIDCOO} & $x$-coordinate of centre of grid box \\
    \texttt{YGRIDCOO} & $y$-coordinate of centre of grid box \\
    \texttt{BOXSIZEX} & half width of grid box in $x$-direction \\
    \texttt{BOXSIZEY} & half width of grid box in $y$-direction \\
    \texttt{COMPLETE} & whether counting has been completed in a grid
    box \\
    \texttt{TOTALCEL} & total number of labelled cell bodies in a grid box \\
    \texttt{TOTALRED} & number of red-labelled cell bodies in a grid box \\
    \texttt{TOTALGRE} & number of green-labelled cell bodies in a grid box \\
    \texttt{TOTALDOU} & number of doubly-labelled cell bodies in a grid box \\
    \texttt{MEANPERI} & \\
    \texttt{MEANAREA} & \\
  \end{tabular}
  \caption{Column headings of the \texttt{SYS.SYS} file.}
  \label{tab:data-format}
\end{table}

The data for each retina is stored in a separate directory. Within
each directory there are two files:
\begin{description}
\item[\texttt{SYS.SYS}] A table in SYSTAT format containing the
  coordinates of the red, green and doubly labelled cell bodies, and
  counts of labelled cell bodies within each grid box. The column
  headings shown in Table~\ref{tab:data-format}.  Each row of the
  table contains information only on a subset of the data, e.g.\ the
  coordinates of a red-labelled cell.
\item[\texttt{ALU.MAP}] A text file containing the coordinates of the
  map outline. The file comprises a number of sections, each starting
  with a single number, which is the number of lines to read in the
  next section. These lines have two numbers each, the $x$ and $y$
  coordinates of a vertex of the map outline.
\end{description}

\section{Reading in and displaying data}
\label{manual:sec:datafile-utils}

All code is to be found in the \texttt{trunk/src} directory. The R
program should be started from this directory in the following examples.

To read in data use the functions in \texttt{datafile-utils.R}. Here
is code to read in the data from a directory contaning the
\texttt{SYS.SYS} and \texttt{ALU.MAP} files, as detailed above:
\begin{verbatim}
source("datafile-utils.R")
sys <- read.sys("/data/path/gm257-1-P8-C57BL6/")
map <- read.map("/data/path/gm257-1-P8-C57BL6/")
plot.sys.map(sys, map)
\end{verbatim}


\end{document}

%%% Local Variables: 
%%% TeX-PDF-mode: t
%%% End: 
