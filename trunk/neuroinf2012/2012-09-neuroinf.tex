\documentclass[final]{beamer} 
\mode<presentation> {  %% check http://www-i6.informatik.rwth-aachen.de/~dreuw/latexbeamerposter.php for examples
  \usetheme{Sterratt}    %% you should define your own theme e.g. for big headlines using your own logos 
}
% \usepackage[garamond]{mathdesign}
\usepackage[english]{babel}
\usepackage[latin1]{inputenc}
\usepackage[T1]{fontenc}
% \usepackage{amsmath,amsthm, amssymb, latexsym}
%\usepackage{helvet}%\usefonttheme{professionalfonts}  % times is
% obsolete

%\renewcommand{\rmdefault}{ugm}
%\renewcommand{\sfdefault}{ugm}
\newcommand{\figack}[1]{{\par\small\vskip -0.5ex\hfill{\color{blue} #1}\par}}
\usepackage{natbib}
\makeatletter
\def\newblock{\beamer@newblock}
\makeatother 


% \usefonttheme[onlymath]{serif}
\boldmath
\usepackage[orientation=portrait,size=a4,debug]{beamerposter}                       % e.g. for DIN-A0 poster
% \usepackage[orientation=portrait,size=a1,scale=1.4,grid,debug]{beamerposter}                  % e.g. for DIN-A1 poster, with optional grid and debug output
% \usepackage[size=custom,width=200,height=120,scale=2,debug]{beamerposter}                     % e.g. for custom size poster
% \usepackage[orientation=portrait,size=a0,scale=1.0,printer=rwth-glossy-uv.df]{beamerposter}   % e.g. for DIN-A0 poster with rwth-glossy-uv printer check
% ...
% 


\usepackage[garamond]{mathdesign}
\usefonttheme{serif}
%\usepackage{garamond}

\title{Retistruct: A package to reconstruct flattened retinae}
\author{David C Sterratt$^1$, Daniel Lyngholm$^2$, David J Willshaw$^1$ and Ian D Thompson$^2$}
\institute{$^1$Institute for Adaptive \& Neural Computation, School of
  Informatics, University of Edinburgh\\ $^2$MRC Centre for Developmental
  Neurobiology, King's College London} \date{Jul. 31th, 2007}
\begin{document}

\begin{frame}{} 
  \begin{columns}[T]

    %%%%%%%%%%%%%%%%%%%%%%%%%%%%%%%%%%%%%%%%%%%%%%%%%% 
    \begin{column}{.37\linewidth}

      \begin{block}{Background \& problem}
        The concept of topographic mapping is central to the
        understanding of the visual system at many levels, from the
        developmental to the computational. It is important to be able
        to relate different coordinate systems, e.g. maps of the
        visual field and maps of the retina. Retinal maps are
        frequently based on flat-mount preparations.  These use
        dissection and relaxing cuts to render the quasi-spherical
        retina into a 2D preparation. The variable nature of relaxing
        cuts and associated tears limits quantitative cross-animal
        comparisons.

      \end{block}

      \begin{block}{A solution: retinal reconstruction}
        \includegraphics[width=\linewidth]{folding}     

        Our ``Retistruct'' algorithm reconstructs retinal flat-mounts
        by mapping them into a standard, spherical retinal space.

      \end{block}

      \begin{block}{Algorithm}

        \includegraphics[width=\linewidth]{fig1-retistruct-method}

        Physically-inspired deformation energy function:
        \begin{displaymath}
          e_{\text L}=\sqrt{\frac 1{2|\mathcal{C}|\overline L}\sum _{i\in\mathcal{C}}\frac{(l_i-L_i)^2}{L_i}}
        \end{displaymath}
        $L_i$ and $l_i$ are lengths of corresponding connections
        $i\in\mathcal{C}$ in flattened and spherical retina.
      \end{block}


      \begin{block}{Low \& high deformation reconstructions}

        \includegraphics[width=\linewidth]{fig2-retistruct-low-high}     
        \begin{itemize}
        \item  Algorithm applied to 297 flat-mounted retinae
        \item 288  reconstructed successfully
        \item 7 failed due to, as-yet unresolved, software problems
        \item 2 rejected because of unsatisfactory reconstructions ($e_\mathrm{L}>0.2$)
        \end{itemize}
      \end{block}

    \end{column}

    %%%%%%%%%%%%%%%%%%%%%%%%%%%%%%%%%%%%%%%%%%%%%%%%%% 
    \begin{column}{.37\linewidth}

      \begin{block}{Eye muscle insertions}
        Orient the retina relative to the nictitating membrane and
        compare this to eye muscle insertions.

        \includegraphics[width=\linewidth]{fig5-Muscles}     
      \end{block}

      \begin{block}{Projections to LGN}

        \includegraphics[width=\linewidth]{fig5-retistruct-ipsi.png}     

        Kernel Density Estimates and Kernel Regression estimates
        computed using Fisherian density \citep{Fisher1953}; bandwidth
        determined by cross-validation.
      \end{block}

      \begin{block}{Conversion to visuotopic coordinates}
        \includegraphics[width=\linewidth]{fig6-retistruct-visuotopic.png}     
        
        \begin{itemize}
      \item When retinal distribution of the ipsilateral ventrotemporal
        crescent neurons is transformed into visuotopic space using
        the optic axis coordinates of 64$^{\circ}$ azimuth and
        22$^{\circ}$ elevation \citep{OommenStahl2008}, the
        decussation line lines up with the vertical meridian.

      \item Optic axis pointing to optic disc coordinates of
        60$^{\circ}$ azimuth and 35$^{\circ}$ elevation
        \citep{Drager1978} $\Rightarrow$ visual mismatch between
        decussation lines.
        \end{itemize}

      \end{block}


      \begin{block}{S-opsin distribution}
        \includegraphics[width=\linewidth]{fig8-visuotopic-orientation}     

        Optic axis orientation ($64^{\circ},22^{\circ}$)
        $\Rightarrow$ dorsoventral boundary for S-opsin expressing
        cones closely matches the horizontal meridian
        (\citealp{HaverkampEtal2005}).

      \end{block}



    \end{column}

    %%%%%%%%%%%%%%%%%%%%%%%%%%%%%%%%%%%%%%%%%%%%%%%%%% 
    \begin{column}{.25\linewidth}


      \begin{block}{Deformation statistics}

        \includegraphics[width=\linewidth]{fig3-deformations}     
        
        More residual deformation in younger, more delicate retinae.

      \end{block}

      \begin{block}{Effect of rim angle}
        \includegraphics[width=\linewidth]{fig3-angle}     
        \begin{itemize}
        \item To determine the rim angle of mouse eyes at varying
          stages of development, measure the distance $d_\mathrm{e}$
          from the back of the eye to the front of the cornea and the
          distance $d_\mathrm{r}$ from the back of the eye to the edge
          of the retina.
        \item Rim colatitude (angle measured from the retinal pole)
          $\varphi _0=\arccos (1-2d_{\text r}/d_{\text e})$
        \item Alternative approach: infer, for each retina, the rim
          angle $\hat\varphi _0$ that minimises the deformation.
        \item Maximum decrease in deformation is 19.1\%; mean
          improvement being 7.2\%.
        \item $\Rightarrow$ improvement does not justify adding
          refinement of the rim angle to the algorithm.
        \end{itemize}
      \end{block}

      \begin{block}{Validation using optic disc locations}


        \includegraphics[width=\linewidth]{fig4-retistruct-ods}     

        $\Rightarrow$ the algorithm can estimate the position of a
        point on the intact retina to within 8$^{\circ}$ of arc (3.6\%
        of nasotemporal axis).
      \end{block}

      \begin{block}{Retistruct package}
        \begin{itemize}
        \item Reconstruction and transformation methods
          implemented in R and will be freely available as an R
          package from http://retistruct.r-forge.r-project.org/
        \item Tested on GNU/Linux, but should also work in MacOS and
          Windows.
        \item Data formats: coordinates of data points and retinal
          outline from an in-house camera-lucida setup, or bitmap
          images with an outline marked up in ImageJ ROI format.
        \item GUI interface to facilitate the marking-up of retinae
          and displaying reconstructed retinae.
        \end{itemize}
      \end{block}


      \begin{block}{\large References}
\setlength{\bibsep}{0pt}
        \newcommand{\myshortjournaltitles}{}
        \bibliographystyle{myshort}
        \small
        \renewcommand{\bf}{\small\bfseries}
        \bibliography{../paper/nmf_morph}
        \renewcommand{\bf}{\bfseries}
      \end{block}


      \begin{block}{\large Acknowledgements}
        \small This work was supported by a Programme Grant from the
        Wellcome Trust (G083305) to DJW, IDT, Stephen Eglen and Uwe
        Drescher. DL was supported by a Medical Research Council (UK)
        Studentship. The authors would like to thank Andrew Lowe,
        Stephen Eglen, Johannes J. J.  Hjorth, and Michael Herrmann
        for their very helpful comments throughout this work. We would
        also like to thank Nicholas Sarbicki for helping with the
        analysis of LGN projections and Henry Eynon-Lewis for
        assistance with dissections for muscle-insertion marking and
        S-opsin staining.

      \end{block}




    \end{column}



  \end{columns}



\end{frame}


\end{document}

% LocalWords:  colliculus RashEtal oppo UptoEtal deve Delaunay KCL
% LocalWords:  ANC Herrmann Willshaw Hennig Wellcome mystrings Retistruct MacOS
% LocalWords:  ImageJ DJW IDT Eglen Uwe Drescher DL Hjorth Sarbicki LGN Eynon
% LocalWords:  nictitating visuotopic ventrotemporal OommenStahl decussation
% LocalWords:  Drager HaverkampEtal
