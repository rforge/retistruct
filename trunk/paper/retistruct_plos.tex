% Template for PLoS
% Version 1.0 January 2009
%
% To compile to pdf, run:
% latex plos.template
% bibtex plos.template
% latex plos.template
% latex plos.template
% dvipdf plos.template

\documentclass[10pt]{article}

% amsmath package, useful for mathematical formulas
\usepackage{amsmath}
% amssymb package, useful for mathematical symbols
\usepackage{amssymb}

% graphicx package, useful for including eps and pdf graphics
% include graphics with the command \includegraphics
\usepackage{graphicx}

% cite package, to clean up citations in the main text. Do not remove.
\usepackage{cite}

\usepackage{color} 

% Use doublespacing - comment out for single spacing
%\usepackage{setspace} 
%\doublespacing


% Text layout
\topmargin 0.0cm
\oddsidemargin 0.5cm
\evensidemargin 0.5cm
\textwidth 16cm 
\textheight 21cm

% Bold the 'Figure #' in the caption and separate it with a period
% Captions will be left justified
\usepackage[labelfont=bf,labelsep=period,justification=raggedright]{caption}

% Use the PLoS provided bibtex style
\bibliographystyle{plos2009}

% Remove brackets from numbering in List of References
\makeatletter
\renewcommand{\@biblabel}[1]{\quad#1.}
\makeatother


% Leave date blank
\date{}

\pagestyle{myheadings}
%% ** EDIT HERE **

% FIXME: Remove
\newcommand{\svninfo}{$ $Rev$ $, $ $Date$ $}
\markboth{\svninfo}{\svninfo}

%% ** EDIT HERE **
%% PLEASE INCLUDE ALL MACROS BELOW

%% FIXME: MUST REMOVE BEFORE SUBMITTING TO PLOS
\usepackage{todo}

%% END MACROS SECTION

\begin{document}

% Title must be 150 characters or less
\begin{flushleft}
{\Large
\textbf{Reconstruction of flattened retinae}
}
% Insert Author names, affiliations and corresponding author email.
\\
David C. Sterratt$^{1,\ast}$, Daniel L. Nedergaard$^{2}$, Ian
D. Thompson$^{2}$
\\
\bf{1} Institute for Adaptive and Neural Computation, School of
Informatics, University of Edinburgh, Edinburgh, Scotland, UK
\\
\bf{2} MRC Centre for Developmental Neurobiology, King's College
London, London, UK
\\
$\ast$ E-mail: david.c.sterratt@ed.ac.uk
\end{flushleft}

% Please keep the abstract between 250 and 300 words
\section*{Abstract}

% In retrograde tracing experiments to determine the mapping of
%   connections from the retina to the superior colliculus in mice, a
%   small blob of dye is injected in the superior colliculus and allowed
%   to diffuse retrogradely down the axons of retinal ganglion cells to
%   their cell bodies in the retina. 

In the course of studying the function and development of the visual
system cells in the retina are often labelled, for example to
determine the distribution of particular categories of cell in the
retina, to assess the location of retrograde tracer or to measure the
distribution of a guidance molecule.  Following labelling the retina
is dissected and flattened, and the distribution of labels is measured
in the flattened retina.  In the process of flattening the retina, a
number incisions are made, complicating analysis as some cells that
were close neighbours are now separated by the incisions.  

We present a computational method that overcomes this problem by
inferring where the labels were in the intact retina before
flattening. The input to the algorithm is the line segments of the
flattened retinal outline with incisions and tears marked up by an
expert. The retinal outline is split into triangular elements whose
positions are then transformed so that they lie on a partial sphere
with the expected dimensions of the intact retina.  The transformation
is adjusted so as to minimise the change in the dimensions of each
triangle from the flattened outline.

The algorithm is able to produce a transformation which is
sufficiently good for visualisation. \todo{More about validation} 

We demonstrate the application of the method in a number of
situations. \todo{More about applications here.}

% Please keep the Author Summary between 150 and 200 words
% Use first person. PLoS ONE authors please skip this step. 
% Author Summary not valid for PLoS ONE submissions.   
\section*{Author Summary}

\section*{Introduction}

The function and development of the retina and the mapping of retinal
ganglion cells has been studied extensively in recent years. In this
field a common experimental procedure is to label cells in the retina
\emph{in vivo} and then dissect and flatten the retina to produce a
whole-mount in which the distribution of labels can be measured. In
order to flatten the retina, a number incisions have to be made, and
tearing can occur within the rim of the retina or the incisions.

Labelled retinal whole-mounts have been used to determine the
distribution of mosaics of particular categories of cell in the
retina \cite{WassBoyc91func,RaveEtal03dete}.  Nearest neighbour
analyses are often used to quantify the regularity of mosaics, but
these are susceptible to boundaries \cite{Cook96spat}, both of the rim
of the retina and those introduced by the incision required during
dissection. These extra boundaries reduce the effective number of
cells that can be analysed, which could be significant in low density
mosaics. 

In the study of topographic mapping, the locations of retrograde
tracers injected in the superior colliculus have been measured in
retinal whole mounts \cite{RebeEtal04rela,RashEtal05oppo}. The
distribution of tracer can be split across incisions and tears,
complicating nearest-neighbour analyses and in some cases making it
impossible to determine a mean location that lies with in the outline
of the flattened retina.

Also in the field of topographic mapping, retinal whole-mounts
labelled with markers for the EphA receptor \cite{ChenEtal95comp} have
been produced. The principal axis of variation in the flat mount
appears to be the naso-temporal axis.  Whole-mounts stained for EphB
receptors have also been produced \cite{BirgEtal00kina} and here the
principal axis is appears to be the dorso-ventral axis. In both cases,
the pixel intensity has been quantified along along the apparent
principal axis but not throughout the rest of the retina.  Together
with the finding that the mapping from a flat screen to the superior
colliculus is approximately Cartesian \cite{DragHube76topo} this
apparent alignment of EphA and EphB gradients along the naso-temporal
and dorso-ventral axes of the retina has lead to the mapping from the
retina to the colliculus being described as Cartesian
\cite{BeviEtal11gene}. However, this cannot be the case as the surface
of the intact retina is by no means a Cartesian surface that can be
described using rectilinear coordinates.  Rather it needs to be
described using curvilinear coordinates. Being able to describe the
density of marker as a function of location in the intact retina would
force a deeper understanding of how the mapping process occurs.

% The locations of the labels are measured in the two-dimensional
% Euclidean space defined by the slide, but this coordinate system is
% not entirely satisfactory for at least two reasons. Firstly, the
% incisions and tears may splits clusters of labelled cells which were
% centred around the same point in the intact retina. The mean location
% of such a split cluster in the Euclidean plane may lie outside the
% retinal outline. Secondly, it is difficult to infer the location with
% respect to the retinal pole and the nasal, temporal, ventral and
% dorsal poles with any degree of accuracy.

Therefore the aim of the method presented in this paper is to infer
the coordinates in intact retinal space of points on a flattened
retina. At present, the retina is approximated as a partial sphere,
and the coordinate system can be thought of as lines of latitude and
longitude on the globe. However, it would be possible to generalise
the method to deal with a retina modelled by any shape with axial
symmetry. The method requires that an expert mark up the cuts and
tears, and also that they indicate the position of at least one
landmark on the rim of the eye. The method has been implemented as a
program written in R, and can reconstruct a retina within a few
minutes on a desktop machine. The accuracy of the method is estimated
to be \todo{produce estimate for accuracy of method}.

% The algorithm is able to produce a transformation which is
% sufficiently good for visualisation. \todo{More about validation} 

We demonstrate the application of the method in a number of
situations. \todo{More about applications here.}


% Results and Discussion can be combined.
\section*{Results}

\subsection*{Examples of the method in action}

\todo{Perhaps present two reconstructed retinae, one ``good'' and one
  ``bad''. Present the strain map and also the measure of goodness.}

\subsection*{Application: retrograde tracing}

\todo{Need to discuss this with Dan and Ian}

\subsection*{Application: mosaic properties?}

\todo{Could discuss this with Stephen}

\subsection*{Application: marker distribution in spherical coordinates}

\todo{This would need some decent whole-mount images of EphA or EphB
  and some more work.}

\section*{Discussion}

\todo{This is all at brain dump stage.}

\subsection*{Approaches to similar problems in the literature}
\label{fold-sphere:sec:appr-simil-probl}

As far as we are aware, a computational algorithm for reconstructing
retinae has never been described before. However, the problem of
mapping 2D between and 3D surfaces appears in other situations, such
as in clothes design and computer animation
\cite{FanEtal98spri,MaCaEtal99flat,WangEtal02surf}. 3D surfaces are
mapped to 2D surfaces by first triangulating the 3D surface, then
projecting the points to the 2D plane, and then moving the locations
of nodes of the triangulation on the plane so as to minimise an
energy function based on the differences between the lengths of edges
in the 3D configuration and in the 2D configuration. A penalty term
to the error function can be added to prevent nodes crossing over
edges \cite{WangEtal02surf}. In the mapping from 2D surface to 3D
surfaces, the effects of forces due to sewing together edges of cloth
and of gravity may be incorporated \cite{FanEtal98spri}.

The basic physical model of the material as a mesh of masses connected
by springs. This model should ensure that neighbourhood relations are
preserved -- nodes close in the 2D structure should still be close in
the 3D structure -- and incorporates the notion of an elastic
material. However, a more physically-principled method of modelling
deformation of objects is offered by the finite element method (FEM)
in which the stress and strain of each element in the mesh is derived
as a function of the deformation of the points of the element
\cite{ZienTayl00fini}.

The FEM is used widely to describe properties of soft tissue in
simulations of surgery \cite{CartEtal05appl}. The soft tissue is
often described as a linear-elastic medium, in which the stress
depends linearly of the strain, but it can also be modelled as a
viscoelastic medium, in which the stress depends on the history of the
strain in the material. 

% \subsection*{Potential problems with the method}
% \label{fold-sphere:sec:appr-simil-probl}

% The bulk viscoelastic properties of bovine retinal tissue have been
% measured.  The Young's modulus of rabbit retina in varying stages of
% development, under the assumption that the retina had linear
% elasticity and a Poission ratio of 0.5 \cite{ReicEtal91deve}.  The
% Poission's ratio is an index of how much a material shrinks in the two
% dimentions perpendicular to an expansion. A perfectly incopressible
% matierial has an Poission's ratio of 0.5. They found that the modulus
% increased threefold between P2 and P15. They also found that the
% stiffness of the retinal tissue depended on its location within the
% retina. The centre of the retina was stiffer than the periphery. This
% was due to the thickness of the tissue, which was greater in the
% centre than in the periphery.  The elastic properties of bovine
% retinal tissue dominate over the viscous properties, and that the
% material has a Poisson's ratio of around 0.45 \cite{LuEtal06visc}.

\subsection*{Alternative methods}
\label{fold-sphere:sec:alternative-methods}

In the process of developing the solution presented in this paper, a
number of options were evaluated:

\paragraph{Forward or backward model:} 

In principle, we could use the FEM, along with data (or assumptions)
about the material properties of retinal tissue to model directly the
action of the retinal being flattened following the dissection
cuts. This would lead to a prediction of how a retina cut in a
particular set of loci would look when flattened. Unfortunately, we do
not have precise knowledge of where the cuts were made in the intact
retina or how tears developed.

It is conceivable that the result of a forward model could be compared
to the flattened outline in question, and the difference between the
two used to modify the locations of tears and cuts on the intact model
retina and that this procedure could be iterated until the patterns of
cuts and tears on the intact retina converged.

However, because it is highly likely that the model is a
simplification of material properties of the real retina, it would be
very unlikely that the outline of flattened model retinal would match
the actual flattened retina, and we would be left with the problem of
how to match up the edges of the actual and model flattened
retinae. It is also difficult to envision an algorithm for inferring
the locations of the tears and cuts that would be efficient. For these
reasons, we rejected an iterative forward-backward modelling approach
in which the tear locations on the intact retina were inferred.

In the simple backward approach undertaken in this paper, the stresses
and strains in the flattened retina are all assumed to be zero. In the
mapping onto the intact retina, the elements of the mesh become
strained. This is clearly unphysical, but may be a reasonable
approximation to the actual Physics because:
\begin{enumerate}
\item For small stresses and strains, if the nodes are mapped back
  onto the flattened retina, the locations shouldn't change much
  (evidence?).
\end{enumerate}

\paragraph{Finite element versus spring-mass}
\label{fold-sphere:sec:finite-elem-vers}

The algorithm presented in this paper models the retinal tissue as a
mesh of masses connected by springs. While this model is used in
modelling deformable surfaces in computer graphics and has been used
in models of tissue, the finite element method is based more strongly
on the physics of materials and is regarded as superior
\cite{CartEtal05appl}. In principle the finite element method allows
parameters of the material to be measured and incorporated in the
model in a straightforward way. Given enough data about
the material properties of retinal tissue, the model could be made
more realistic. 

Weighed against this, is the fact that it is not possible to apply the
finite element method simply to the problem of mapping the flattened
retina onto the intact retina, as this would require knowledge of the
stresses and strains in the flattened retina, whereas the only
information available is the outline of the retina and the location of
the optic disk. 

Nevertheless, it would be possible to attempt to try applying the
finite element, initialised with unstressed elements, to the backward
transformation. This would give an estimate of the positions of the
locations of nodes on the intact retina. These nodal positions could
then be used to estimate the stiffness matrix for each element,
assuming that each element was unstressed on the intact retina. The
deformation of this structure when flattened could then be estimated
by fixing the locations around the rim and edges, and solving for the
unknown positions. This would lead to each element being strained, and
these strains could then be used to initialise a second mapping back
on to the intact retina, which would hopefully give rise to a more
accurate estimation of the nodal positions. 

This method would have the advantages of greater physical plausibility,
and would afford the opportunity to incorporate elastic properties
which vary with distance from the centre of the retina. However, this
would come at the cost of some increase in complexity.

\paragraph{Supervised versus unsupervised cut and tear detection}
\label{fold-sphere:sec:superv-vers-unsup}

\begin{itemize}
\item Unsupervised reconstruction, with edge binding.  I envisage that
  it would be possible to put in a component of the energy function
  that provides for short range repulsion between vertices on the edge
  of the flattened retina and edges of the flattened retina. The
  energy function might have longer range attraction. One possible
  parametrisation would be the Lennard-Jones potential which is used
  to model short range repulsion and longer range attraction between
  molecules:
  \begin{equation}
    \label{fold-sphere:eq:4}
    E(r) = 4\epsilon\left(\left(\frac{\sigma}{r}\right)^{12}-
      \left(\frac{\sigma}{r}\right)^{6}\right)
  \end{equation}
  where $r$ is the distance between molecules, $\epsilon$ defines the
  depth of the potential well and $\sigma$ is the distance at which the
  potential is zero.

  Were this approach to work, additional refinements might be possible,
  such as adding a component of the energy function that draws together
  points on either side of a rip which are correspond to each other with
  high probability. Another possibility would be to allow the radius of
  the sphere $R$ to vary within realistic bounds, to optimise the fit
  further.
\item Automated stitching.
\end{itemize}


% You may title this section "Methods" or "Models". 
% "Models" is not a valid title for PLoS ONE authors. However, PLoS ONE
% authors may use "Analysis" 
\section*{Materials and Methods}
\label{retistruct_plos:sec:materials-methods}

\subsection*{Retinae}
\label{retistruct_plos:sec:retinae}

\todo{Some information about the preparation of the retinae.}

\subsection*{Reconstruction algorithm}
\label{retistruct_plos:sec:reconstr-algor}

The reconstruction algorithm described here has been implemented in R
(http://www.r-project.org) and can be downloaded as an R package from
http://www.neuralmapformation.org/code/retistruct. It has been tested
under a GNU/Linux platform but should also work in MacOS or Windows
environments.

The steps of the reconstruction algorithm are illustrated in
Figure~\ref{fold-sphere:fig:method}. The raw data
(Figure~\ref{fold-sphere:fig:method}A) consists of the sequence of
points making up the outline, sets of data points and sequences of
connected points defining landmarks, such as the optic disk. The
reconstruction process then proceeds as follows:
\begin{enumerate}
\item The location of one of the poles and incisions and tears in the
  outline are marked up by an expert
  (Figure~\ref{fold-sphere:fig:method}B). Each tear is defined by a
  central point, referred to as the apex, and two outer points,
  referred to as vertices.  Tears within tears or incisions (for
  example tear 2 in Figure~\ref{fold-sphere:fig:method}B) can be
  marked up. The Retistruct package contains a graphical user
  interface to allow incisions and tears to be marked up quickly.
\item The retinal outline is triangulated using the conforming
  Delaunay triangulation algorithm provided by the Triangle package
  \cite{Shew96tria} such that there are at least 500 triangles in
  the outline (grey lines in Figure~\ref{fold-sphere:fig:method}C).
\item The tears and incisions are stitched automatically (blue and
  green lines in Figure~\ref{fold-sphere:fig:method}D). To do this,
  the length of each side of every tear is computed. The fractional
  distance of each point in each tear is then defined as the distance
  along the tear of that point from the apex divided by the total
  length of that side of the tear. For each point on one side of a
  tear a point is inserted at the same fractional distance along the
  opposing side. Tears within tears are dealt with by excluding the
  child tear when computing the fractional distance. At the end of the
  procedure there is a set of correspondences between two or, in the
  case of the vertices of a child tear, three points.
\item There is then an extra round of triangulation to incorporate the
  points that have been inserted into incisions and tears.
\item The points within each set of correspondences are merged and
  allocated positions on the 2D surface.
\item The triangulation points are then projected roughly onto a
  sphere curtailed at latitude of $\phi_0$
  (Figure~\ref{fold-sphere:fig:method}D). The latitude is estimated
  on the basis of measurements from intact retinae of animals of the
 same age as the retina under reconstruction. The radius $R$ of the
  sphere determined by area of the flattened retina and $\phi_0$.
  Points on the rim of flattened retina are fixed to the rim of the
  curtailed sphere. 
\item The optimal projection onto sphere
  (Figure~\ref{fold-sphere:fig:method}E) is inferred by changing
  location of vertices on sphere so as to minimise an error
  measure. This measure, $E$, comprises the sum of normalised squared
  differences between lengths of corresponding connections in
  flattened and spherical retina and a penalty term that prevents the
  triangles from flipping over:
  \begin{equation}
    E = \frac{1}{2|\mathcal{C}|\overline{L}} \sum_{i\in\mathcal{C}} \frac{(l_i - L_i)^2}{L_i}  
    + \alpha\sum_{j\in\mathcal{T}} f(a_j/A_j)
  \end{equation}
  where $L_i$ and $l_i$ are lengths of corresponding edges
  $i\in\mathcal{C}$ in the flattened and spherical retina
  respectively, $\overline{L}$ is the mean length of an edge,
  $|\mathcal{C}|$ is the number of edges, $\alpha$ is a constant, $f$
  is a penalty function to be defined below, and $A_j$ and $a_j$ are
  signed areas of corresponding triangles $j\in\mathcal{T}$ in the
  flattened and spherical retina.  The signed area $a_i$ is positive
  for triangles in correct orientation, but negative for flipped
  triangles. The penalty function $f$ is a piecewise, smooth function
  that increases steeply with negative arguments:
  \begin{equation}
    \label{retistruct_plos:eq:1}
    f(x) = \left\{
        \begin{array}{ll}
          -(x - x_0/2) & x < 0 \\
          \frac{1}{2x_0}(x - x_0)^2 & 0 < x <x_0 \\
          0 & x \ge x_0
          \end{array} \right.
  \end{equation}
  The parameter $x_0$ is set at 0.1. There is thus no penalty unless
  triangles have been squashed to less than 10\% of their size in the
  flattened retina.  The parameter $\alpha$ is set to be large enough
  to prevent flipped triangles without causing numerical problems with
  the optimisation. The length of each edge $l_i$ is computed from the
  formula for the central angle between its vertices:
  \begin{equation}
    \label{retistruct_plos:eq:2}
    l(\phi_1, \lambda_1, \phi_2, \lambda_2) =
    R(\cos\phi_1\cos\phi_2\cos(\lambda_1-\lambda_2) +
    \sin\phi_1\sin\phi_2)
  \end{equation}
  where $\phi_1$ and $\phi_2$ are the latitudes of the vertices and
  $\lambda_1$ and $\lambda_2$ are the longitudes.  The derivatives of
  $E$ with respect to $\phi_i$ and $\lambda_i$ are computed and the
  BFGS quasi-Newton method implemented in the R optim function is used
  to minimise $E$.
\item The locations of data points and landmarks on the sphere are
  determined (Figure~\ref{fold-sphere:fig:method}F). To do this, for
  each data point the its barycentric coordinates within its containing
  triangle in the flat representation are determined.  The location of
  the point on the sphere is then found by projecting the point to the
  same set of barycentric coordinates in the corresponding triangle on
  the sphere. From this location in Cartesian 3D space, the spherical
  coordinates of the point are determined by projection of a line from
  the centre of the sphere through the point to the line's
  intersection with the sphere. This allows plotting of points in a
  polar representation. The procedure can be used in reverse to infer
  the locations of lines of latitude and longitude in the flat
  retina.
\item The Karcher mean of each set of points is determined
  (Figure~\ref{fold-sphere:fig:method}F). The Karcher mean of a set of
  points on the sphere \cite{Karc77riem,HeoSmal06form} is defined as
  the point $(\overline{\phi}, \overline{\lambda})$ that has a minimal
  sum of squared distances to the set of points $(\phi_i, \lambda_i)$:
  %% See also BergWerm06
  \begin{equation}
    \label{retistruct_plos:eq:3}
    (\overline{\phi}, \overline{\lambda}) = \mbox{arg min}_{(\phi,
      \lambda)} \sum_{i=1}^N l^2(\phi, \lambda, \phi_i, \lambda_i)
  \end{equation}
  where the function $l$ is defined in
  Equation~(\ref{retistruct_plos:eq:2}).
\end{enumerate}

The procedure of reconstructing a retina takes a few minutes on a
modern desktop computer.

% Do NOT remove this, even if you are not including acknowledgments
\section*{Acknowledgments}

David Willshaw and Stephen Eglen (Perhaps should go on the author list?).

%\section*{References}
% The bibtex filename
\newcommand{\myshortjournaltitles}{}
\bibliography{nmf_morph}
%\bibliography{template}

\section*{Figure Legends}
%\begin{figure}[!ht]
%\begin{center}
%%\includegraphics[width=4in]{figure_name.2.eps}
%\end{center}
%\caption{
%{\bf Bold the first sentence.}  Rest of figure 2  caption.  Caption 
%should be left justified, as specified by the options to the caption 
%package.
%}
%\label{Figure_label}
%\end{figure}

\begin{figure}[!ht]
  % Example. e.g. GM114-4/R-CONTRA (One subtear + gap)
  % GM184-5/R-CONTRA (One subtear, butter gap)
  % GM263-1/R-CONTRA (perfect!)
  \includegraphics{retistruct-method}
  
  \vspace*{-4.54in}

  \mbox{\hspace{2.27in}{
      \includegraphics[width=2.27in]{final-projection}
      \includegraphics[width=2.27in]{initial-projection}
    }}

  \vspace*{2.27in}

  \caption{\textbf{Overview of the method.} \textbf{(A)} The raw data:
    a retinal outline (black), three types of data point (red, green
    and yellow circles) and a landmark (blue line). \textbf{(B)}
    Retinal outline with nasal pole (N) and tears marked up. Each
    pair of red and orange lines indicate the vertices and apex of the
    five tears. Note that tear 2 is a child tear of tear 1. \textbf{(C)}
    The outline after triangulation (shown by grey lines) and
    stitching, indicated by blue lines between corresponding points
    on the tears. Green lines indicate connections between tear
    vertices.  \textbf{(D)} The initial projection of the triangulated
    and stitched outline onto a hemisphere. Tears are show in
    white. Below is a representation of lines of latitude and
    longitude on the flat outline. \textbf{(E)} The projection of the
    flat retina onto the sphere and the lines of latitude and
    longitude after optimisation of the mapping. \textbf{(F)} The data
    represented on a polar plot of the reconstructed retina. Mean
    locations of the different types of data points are indicated by
    filled diamonds. The nasal (N), dorsal (D), temporal (T) and
    ventral (V) poles are indicated. For reference, data is plotted on
    the flat representation below. }
  \label{fold-sphere:fig:method}
\end{figure}



\section*{Tables}
%\begin{table}[!ht]
%\caption{
%\bf{Table title}}
%\begin{tabular}{|c|c|c|}
%table information
%\end{tabular}
%\begin{flushleft}Table caption
%\end{flushleft}
%\label{tab:label}
% \end{table}

\end{document}


% LocalWords:  Sterratt Nedergaard MRC MacOS Retistruct BFGS optim Karcher arg
% LocalWords:  Acknowledgments nmf naso EphB dorso
