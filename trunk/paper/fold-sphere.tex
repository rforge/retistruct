\documentclass{article}

\newcommand{\svninfo}{$ $Rev$ $, $ $Date$ $}
\pagestyle{myheadings}
\markboth{\svninfo}{\svninfo}

\usepackage{xspace}
\usepackage{amsmath}
\usepackage{graphicx}
\usepackage{natbib}
\usepackage{todo}
\usepackage[british]{babel}
\usepackage[left=1in,right=1in,top=1in,bottom=1in]{geometry}

\newcommand{\EE}{\ensuremath{E_\mathrm{E}}\xspace}
\newcommand{\EA}{\ensuremath{E_\mathrm{A}}\xspace}
\newcommand{\ED}{\ensuremath{E_\mathrm{D}}\xspace}
\newcommand{\Ed}{\ensuremath{E_\mathrm{d}}\xspace}
\newcommand{\p}{\vec{p}}
\newcommand{\q}{\vec{q}}


\title{Inference of original retinal coordinates from  flattenend
  retinae}
\author{David C. Sterratt, Daniel L. Nedergaard and Ian D. Thompson}

\begin{document}
\maketitle
\thispagestyle{myheadings}

\begin{abstract}
  In retrograde tracing experiments to determine the mapping of
  connections from the retina to the superior colliculus in mice, a
  small blob of dye is injected in the superior colliculus and allowed
  to diffuse retrogradely down the axons of retinal ganglion cells to
  their cell bodies in the retina. The retina is dissected and
  flattened, and the pattern of dye in cell bodies can be seen in the
  flattened retina.  In the process of flattening the retina, a number
  incisions are made, and the pattern of dye can cut across incisions,
  complicating analysis.  One way of simplifying the analysis would be
  to infer the position of the cell bodies in the spherical coordinate
  system of the intact retina.  
\end{abstract}

\section{Introduction}
\label{fold-sphere:sec:introduction}

There are a number of experiments in which the retina is labelled,
then dissected, flattened, and the locations of labels analysed
\citep{RashEtal05oppo} \todo{More examples of analysis of retinae}.
During the dissection, incisions are made in retina, and tearing may
occur as the retina is flattened. The locations of the labels are
measured in the two-dimensional Euclidean space defined by the slide,
but this coordinate system is not entirely satisfactory for at least
two reasons. Firstly, the incisions and tears may splits clusters of
labelled cells which were centred around the same point in the intact
retina. The mean location of such a split cluster in the Euclidean
plane may lie outside the retinal outline. Secondly, it is difficult
to infer the location with respect to the retinal pole and the nasal,
temporal, ventral and dorsal poles with any degree of accuracy.

Therefore the aim of the method presented in this paper is to infer
the coordinates in intact retinal space of points on a flattened
retina. At present, the retina is approximated as a partial sphere,
and the coordinate system can be thought of as lines of latitude and
longitude on the globe. However, it would be possible to generalise
the method to deal with a retina modelled by any shape with axial
symmetry.  The method requires that an expert mark up the cuts and
tears, and also that they indicate the position of at least one
landmark on the rim of the eye. The method has been implemented as a
program written in R, and can reconstruct a retina within a few
minutes on a desktop machine. The accuracy of the method is estimated
to be \todo{produce estimate for accuracy of method}.

\todo{Previous work??} The authors are not aware of this particular problem
having been addressed before, although the  finite element method for
modelling materials is in common use in other fields \todo{Citations
  of where FEM is used in other fields.}

% The general idea behind this method is to imagine folding the
% flattened retina onto a manifold with the same shape and size of the
% real eye. In order to perform the folding, the method has to allow for
% some elasticity in the retina. The method as developed so far does
% incorporate elasticity, but the prevention of overlap has not yet been
% implemented.

\section{Results}
\label{fold-sphere:sec:results}

Figures~\ref{fold-sphere:fig:test1} and~\ref{fold-sphere:fig:test2}
show that the algorithm does project the flattened retina onto a
sphere.

\subsection{Validation}
\label{fold-sphere:sec:validation}



\section{Discussion}
\label{fold-sphere:sec:discussion}

\subsection{Validity of the method}
\label{fold-sphere:sec:validity-method}

\subsection{Alternative methods}
\label{fold-sphere:sec:alternative-methods}

\begin{itemize}
\item Unsupervised reconstruction, with edge binding.  I envisage that
  it would be possible to put in a component of the energy function
  that provides for short range repulsion between vertices on the edge
  of the flattened retina and edges of the flattened retina. The
  energy function might have longer range attraction. One possible
  parametrisation would be the Lennard-Jones potential which is used
  to model short range repulsion and longer range attraction between
  molecules:
  \begin{equation}
    \label{fold-sphere:eq:4}
    E(r) = 4\epsilon\left(\left(\frac{\sigma}{r}\right)^{12}-
      \left(\frac{\sigma}{r}\right)^{6}\right)
  \end{equation}
  where $r$ is the distance between molecules, $\epsilon$ defines the
  depth of the potential well and $\sigma$ is the distance at which the
  potential is zero.

  Were this approach to work, additional refinements might be possible,
  such as adding a component of the energy function that draws together
  points on either side of a rip which are correspond to each other with
  high probability. Another possibility would be to allow the radius of
  the sphere $R$ to vary within realistic bounds, to optimise the fit
  further.
\item Automated stitching.
\end{itemize}

\subsection{Applications of the method}
\label{fold-sphere:sec:applications-method}


\section{Algorithm}
\label{fold-retina:sec:method}



\subsection{Triangulation of flattened retina}
\label{fold-sphere:sec:triang-flatt-retina}

\begin{figure}[tp]
  \centering
  %\includegraphics{mesh}
  \caption{Triangular mesh}
  \label{fold-sphere:fig:mesh}
\end{figure}

The edge of the flattened retina is described by a set of $N$ points:
$\mathcal{P} = \{\vec{p}_1,\dots,\vec{p}_N$\}. Connections between
points are represented by forwards and backwards pointers, which are
represented as vectors $\vec{g}^+$ and $\vec{g}^-$:
\begin{displaymath}
  g^+_i = \left\{
    \begin{array}{ll}
      i+1 & i < N \\
      1   & i = N
    \end{array}\right.
  \quad
  g^-_i = \left\{
    \begin{array}{ll}
      i-1 & i > 1 \\
      N   & i = 1
    \end{array}\right.
\end{displaymath}

A triangular mesh with the set of points $\mathcal{P}$ as its boundary
is created using the conforming Delaunay triangulation algorithm
provided by the Triangle package \citep{shewchuk96b}. By summing the
areas of the triangles in this rough triangulation, the total area $A$
of the retina is computed. The retina is then triangulated again into
at least $n=200$ triangles, by using the conforming Delaunay
triangulation with the additional constraint that each triangle should
by no larger than $A/n$, and and with Ruppert's
\todo{Reference for Ruppert} method for ensuring the smallest angle of
any triangle is no less than $20^\circ$.

The triangulation comprises $M$ triangles and is described by an
$M\times3$ matrix $\mathbf{T}$, where each row contains the indicies
of the points in the triangle in anticlockwise order.

The area $a_i$ of each triangle is computed:
\begin{displaymath}
  a_i = |0.5 (\vec{p}_{T_{i2}}-\vec{p}_{T_{i1}})\times
  (\vec{p}_{T_{i3}}-\vec{p}_{T_{i2}})|
\end{displaymath}

\begin{enumerate}
\item From the triangulation $\mathbf{T}$ the set $\mathcal{C}$ of pairs of
  indicies representing line segments is constructed.
\item For the morphing algorithm, it is essential that no line segment
  connects non-adjacent nodes in the rim $\mathcal{R}$. Such line
  segments may be created by the triangulation. To detect such line
  segments, each member of $\mathcal{C}$ is checked to see if it is a
  subset of $\mathcal{R}$ and that its ends are not adjacent, by
  using $g^+$ and $g^-$. For each line segment that is found, it is
  removed by
  \begin{itemize}
  \item removing the two triangles that it belongs to from the list
    $\mathbf{T}$
  \item Creating a new point at the centroid of the four vertices
    shared by these two triangles. This new point $i$ is added to the
    correspondences vector: $h_i=i$.
  \item Creating four new triangles, who all share the new point and
    the each of which has two vertices from the set of four
    triangles.
  \end{itemize}
\end{enumerate}

The connection set is converted into $\mathbf{C}$, a symmetric,
binary-valued matrix that defines if there is a connection between $i$
and $j$.  

The distances $L_{ij}$ are between grid points on the flattened retina
are computed.

\subsection{Stitching algorithm}
\label{fold-sphere:sec:stitching-algorithm}

\begin{figure}[tp]
  \centering
  %\includegraphics{stitch}  
  \caption{Stitching. The solid line is the outline of the retina. The
    red circled-points are ones that were originally on the rim. The
    blue lines are the + tears and the purple lines are the - tears.
    Green lines indicate corresponding vertices at the end of tears.
    Yellow lines indicate corresponding points along the tears. }
  \label{fold-sphere:fig:stitch}
\end{figure}

Figure~\ref{fold-sphere:fig:stitch} shows a flattened
retina which has had the stitching algorithm applied to it. 

Cuts and tears marked up by an expert.  Each of the $M$ cuts or tears
$j$ is defined by a common apex at the point indexed by $A_j$, a
vertex in the forwards direction (indexed by $V^+_j$) and in the
backwards directions (indexed by $V^-_j$).  Pairs of vertices will
correspond to each other in the morphed retina, and this relationship
is indicated by a correspondence vectors $\vec{h}^+$ and $\vec{h}^-$:
\begin{displaymath}
  h^+_i =  \left\{
    \begin{array}{ll}
      i & \mbox{ if } i \not\in \{V^-_1,\dots, V^-_M\} \\
      V^+_j  & \mbox{ if } \exists j: i = V^-_j
    \end{array}\right.
  \quad
  h^-_i =  \left\{
    \begin{array}{ll}
      i & \mbox{ if } i \not\in \{V^+_1,\dots, V^+_M\} \\
      V^-_j  & \mbox{ if } \exists j: i = V^+_j
    \end{array}\right.
\end{displaymath}
At this stage, a correspondence vector $\vec{h}$ is initialised to be
the same as $\vec{h}^+$.

The set of points in each tear is determined using the function
$\mathrm{path}$:
\begin{displaymath}
  \mathrm{path}(i, j, \vec{g}, \vec{h})  = \left\{ 
  \begin{array}{ll}
    \{i\} & \mbox{ if } i = j \\
      \{i, \mathrm{path}(g_i, j, \vec{g}, \vec{h})\} & \mbox{ if } i \ne j, h_i=i \\
      \{i, \mathrm{path}(h_i, j, \vec{g}, \vec{h})\}    & \mbox{ if } i \ne j, h_i\ne i \\
    \end{array}\right.
\end{displaymath}
This allows us to write:
\begin{displaymath}
  \mathcal{T}^+_j  = \mathrm{path}(A_j, V_j^+, \vec{g}^+, \vec{h}^+) \quad 
  \mathcal{T}^-_j  = \mathrm{path}(A_j, V_j^-, \vec{g}^-, \vec{h}^-)
\end{displaymath}
It is useful to determine the set of points $\mathcal{R}$ on the rim
of the retina.
\begin{displaymath}
  \mathcal{R} = \{1,\dots,N\} 
  \setminus (\mathcal{T}^+_1 \setminus V^+_1) 
  \setminus (\mathcal{T}^-_1 \setminus V^-_1)  
  \dots 
  \setminus (\mathcal{T}^+_M \setminus V^+_M)
  \setminus (\mathcal{T}^-_M \setminus V^-_M)
\end{displaymath}

The function $\mathrm{pl}$ defines the path length from one point to another
point:
\begin{displaymath}
  \mathrm{pl}(i, j, \vec{g}, \vec{h}, \mathcal{P}) = \left\{ 
    \begin{array}{ll}
      0 & \mbox{ if } i = j \\
      |\vec{p}_i-\vec{p}_{g_i}| + \mathrm{pl}(g_i, j, \vec{g}, \vec{h}, \mathcal{P}) & \mbox{ if } i \ne j, h_i=i \\
      \mathrm{pl}(h_i, j, \vec{g}, \vec{h}, \mathcal{P})    & \mbox{ if } i \ne j, h_i\ne i \\
    \end{array}\right.
\end{displaymath}
For each tear, the length of each side of the tear is computed:
\begin{displaymath}
  S^+_j = \mathrm{pl}(A_j, V^+_j, \vec{g}^+, \vec{h}^+, \mathcal{P}) \quad 
  S^-_j = \mathrm{pl}(A_j, V^-_j, \vec{g}^-, \vec{h}^-, \mathcal{P})
\end{displaymath}
Then for each point $i$ on the $+$ side of the tear for which there is
no corresponding point $h_i^-$, a new, corresponding, point, with the
index $n=N+1$ is placed at the same fractional distance along the
corresponding $-$ tear. To do this, the distance of a point
$i\in{\mathcal{T}^+_j}\setminus A_j \setminus V^+_j$ along the + tear
is computed:
\begin{displaymath}
  s^+_{ji} = \mathrm{pl}(A_j, i, \vec{g}^+, \vec{h}^+, \mathcal{P}) \quad 
\end{displaymath}
The node $k$ in the corresponding tear is the node which has the node
with the highest fractional distance $s^-_{jk}=\mathrm{pl}(A_j, k,
\vec{g}^-, \vec{h}^-, \mathcal{P})$ along the - tear which is still below
$s^+_{ji}$. The location of the new point is
\begin{displaymath}
  \vec{p}_n = (1-f)\vec{p}_k + f\vec{p}_{g^-_k}
\end{displaymath}
where
\begin{displaymath}
  f = \frac{s^+_{ji}S^-_j/S^+_j-s^-_{jk}}{s^-_{jg^-_k}-s^-_{jk}}
\end{displaymath}
The correspondences vector is updated so that $h_i = n$. The vector of
forward pointers is updated so that
\begin{displaymath}
  \begin{array}{ll}
    g^-_n = k     & g^+_n = g^+_k \\
    g^-_{g^+_k} = n & g^+_k = n 
  \end{array}
\end{displaymath}
If a correspondence $h^-_i$ exists for point $i$, then no new point is
made, but this point is set to correspond to the same point as $h^-_i$
corresponds to by setting $h_i = h_{h^-_i}$. 

A similar procedure is carried out for the $-$ tear.

After this procedure, it is possible that there may be chains of
correspondences where, for example, $i$ corresponds to $h_i$ and $h_i$
correspondences to a different point $h_{h_i}$. These three points are in
fact actually one, but to indicate this, both $h_i$ and $i$ should
point to $h_{h_i}$.  It order to ensure this, the correspondence
vector is updated iteratively:
\begin{displaymath}
  h^{k+1}_i = h^k_{h^k_i}  
\end{displaymath}
until all $h^{k+1}_i = h^k_i$. The final $\vec{h}^{k+1}$ is referred
to as $\vec{h}$.

% The stitching algorithm uses this information to create
% correspondances between points on cuts and tears using an algorithm
% that


\subsection{Projection onto hemisphere}
\label{fold-sphere:sec:proj-onto-hemisph}

When the points are projected onto the hemisphere, the points which
have the stitching algorithm has identified as corresponding with each
other are merged. This results in a new set of points, which is a
subset of the existing set of points created by the triangulation and
stitching operations. It is desirable to renumber this set of points
consecutively starting at 1. In order to do this, 
\begin{itemize}
\item An ordered set $\mathcal{U}$ of the unique values of the range
  of $h$ is formed. $\mathcal{U}$ has $\tilde N$ elements. The mapping
  $u: \{1,\dots, \tilde N\} \rightarrow \mathcal{U}$ maps the new index
  to the old index.
\item A transformed set of points $\mathcal{\tilde P}$ is formed by
  setting $\tilde{\vec{p}}_i = \vec{p}_{u_i}$
\item A new mapping $\tilde h$ from the existing indicies to a
  new set of unique indicies is formed:
  \begin{displaymath}
    \tilde h_i = j:u_j = h_i \quad
    \tilde h(i) = u^{-1}(h(u(i)))
  \end{displaymath}
\item A transformed triangulation $\tilde{\mathbf{T}}$ is derived from
  $\mathbf{T}$, by setting $\tilde{T}_{ij} = \tilde h_{T_{ij}}$
\end{itemize}

It is also necessary to transform the indicies used by the edge sets
and determine correspondences between edges which have been merged. 
\begin{itemize}
\item A mapping $\vec{H}$ between connections (as opposed to points)
  is formed by 
\item A transformed set of edges is formed by setting $C'_{ij} =
  \tilde h(C_{ij})$. A number of the rows of this matrix contain
  identical elements to each other. A mapping $H$ from each rows of
  this matrix the corresponding row. This is analogous to the mapping
  $h$ for points. Mappings $\tilde{H}$ and $U$ are formed
  analogously to $\tilde{h}$ and $u$.
\item The final connection matrix $\tilde{\mathbf{C}}$ is formed using
  $\vec{U}$ and $\mathbf{C}'$: $\tilde C_{ij} = C'_{U_ij}$. The number
  of rows in $\tilde{\mathbf{C}}$ is $\tilde M$.
\item New $\tilde L_k = \frac{1}{|\mathcal{H}_k|}
  \sum_{i\in\mathcal{H}_k} L_i$ where $\mathcal{H}_k = i: k=H_k$. 
\item In order to compute derivatives efficiently later on in the
  procedure, a we form a $\tilde N$ by $2\tilde M$ matrix
  $\tilde{\mathbf{B}}$ which maps by setting $\tilde B_{\tilde
    C_{k1}k} = 1$ and $\tilde B_{\tilde C_{k2}k} = 1$ for all
  $k=1\dots \tilde M$.
\end{itemize}

The total area $A$ of the retina is computed by summing the areas of
individual triangles:
\begin{displaymath}
  A = \sum_{i=1}^M a_i
\end{displaymath}
where $a_i = | a_i^\mathrm{sign}|$.  It is supposed that this grid is
to be projected onto a sphere with a radius appropriate for the area
$A$ of the flattened retina. The radius is
\begin{equation}
  \label{fold-sphere:eq:1}
  R = \sqrt{\frac{A}{2\pi\sin\phi_0+1}}
\end{equation}
where $\phi_0$ is the latitude at which the rim of the intact retina
would be expected.

The aim now is to infer the latitude $\phi_i$ and longitude
$\lambda_i$ on the sphere to which each grid point $i$ on the
flattened retina is projected.  It is proposed to achieve this aim by
minimising an energy function which depends on three components, an
elastic component $E_\mathrm{E}$ and an area-preserving component
$E_\mathrm{A}$:
\begin{equation}
  \begin{split}
  E(\phi_1,\dots,\phi_N,\lambda_1,\dots,\lambda_N) = & \\
  E_\mathrm{E}(\phi_1,\dots,\phi_N,\lambda_1,\dots,\lambda_N) 
  & + E_\mathrm{A}(\phi_1,\dots,\phi_N,\lambda_1,\dots,\lambda_N) 
  \end{split}
\end{equation}

\subsection{The elastic energy}
\label{fold-sphere:sec:elastic-force}

This component of the energy corresponds to the energy contained in
imaginary springs with the natural length of the distances in the
flattened retina, $\tilde L_k$:
% \begin{equation}
%   \label{fold-sphere:eq:5}
%   \EE  = \frac{1}{2} \sum_{i=1}^N \sum_{j=1}^N C_{ij} (l_{ij}-L_{ij})^2  
% \end{equation}
\begin{equation}
  \label{fold-sphere:eq:5}
  \EE  = \frac{1}{2} \sum_{k=1}^{\tilde M} \frac{(\tilde l_k- \tilde
    L_k)^2}{\tilde L_k}
\end{equation}
where $\tilde l_k$ is the distance of the $k$th edge between grid points on
the sphere and is given by the formula for the central angle:
\begin{equation}
  \label{fold-sphere:eq:2}
  \tilde l_k = R\cos^{-1}(\cos\phi_i\cos\phi_j\cos(\lambda_i-\lambda_j) + \sin\phi_i\sin\phi_j)
\end{equation}
where $i=\tilde C_{k1}$ and $j=\tilde C_{k2}$.  Minimising this energy function
should lead to the distances between neighbouring points on the sphere
being similar to the corresponding distances on the flattened retina.

In order to minimise the function efficiently, the derivatives with
respect to $\phi_i$ and $\lambda_i$ are found:
\begin{equation}
  \label{fold-sphere:eq:3}
  \begin{split}
    \frac{\partial \EE}{\partial\phi_i} = 
    \sum_j C_{ij} (l_{ij} - L_{ij})R
    \frac{\sin\phi_i\cos\phi_j\cos(\lambda_i-\lambda_j) - \cos\phi_i\sin\phi_j}
    {\sqrt{1-(\sin\phi_i\sin\phi_j +
        \cos\phi_i\cos\phi_j\cos(\lambda_i-\lambda_j))^2}} \\
    \frac{\partial \EE}{\partial\lambda_i} = 
    \sum_j C_{ij} (l_{ij} - L_{ij})R
    \frac{\cos\phi_i\cos\phi_j\sin(\lambda_i-\lambda_j)}
    {\sqrt{1-(\sin\phi_i\sin\phi_j + \cos\phi_i\cos\phi_j\cos(\lambda_i-\lambda_j))^2}}
  \end{split}
\end{equation}
A quasi-Newton-Raphson method is then used in R to achieve this
optimisation, and the resulting grid on the sphere is plotted in 3D
(Figure~\ref{fold-sphere:fig:test2}).

\subsection{The area-preserving energy}
\label{fold-sphere:sec:area-pres-energy}


\bibliographystyle{natbib}
\bibliography{nmf_morph}

\end{document}

% LocalWords:  ij BP Raphson PDF myheadings RashEtal oppo shewchuk Ruppert's ji
%%% Local Variables: 
%%% TeX-PDF-mode: t
%%% End: 
% LocalWords:  jk th nmf
