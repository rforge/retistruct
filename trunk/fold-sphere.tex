\documentclass{article}

\usepackage{xspace}
\usepackage{amsmath}
\usepackage{graphicx}
\usepackage{todo}
\usepackage[british]{babel}
\usepackage[left=1in,right=1in,top=1in,bottom=1in]{geometry}

\newcommand{\EE}{\ensuremath{E_\mathrm{E}}\xspace}
\newcommand{\EA}{\ensuremath{E_\mathrm{A}}\xspace}
\newcommand{\ED}{\ensuremath{E_\mathrm{D}}\xspace}
\newcommand{\Ed}{\ensuremath{E_\mathrm{d}}\xspace}
\newcommand{\p}{\vec{p}}
\newcommand{\q}{\vec{q}}


\title{Inference of original retinal coordinates from  flattenend
  retinae}
\author{David Sterratt}

\begin{document}
\maketitle

\begin{abstract}
  In retrograde tracing experiments to determine the mapping of
  connections from the retina to the superior colliculus in mice, a
  small blob of dye is injected in the superior colliculus and allowed
  to diffuse retrogradely down the axons of retinal ganglion cells to
  their cell bodies in the retina. The retina is dissected and
  flattened, and the pattern of dye in cell bodies can be seen in the
  flattened retina.  In the process of flattening the retina, a number
  incisions are made, and the pattern of dye can cut across incisions,
  complicating analysis.  One way of simplifying the analysis would be
  to infer the position of the cell bodies in the spherical coordinate
  system of the intact retina.  
\end{abstract}

\section{Introduction}
\label{fold-sphere:sec:introduction}

The general idea behind this method is to image folding the flattend
retina onto a sphere (or ellipsoid) that has the same dimensions that
would be expected of the real eye. Tears in the retina will naturally
tend to close up as they are folded onto the sphere.  In order to
perform the folding, the method has to allow for some elasticity in
the retina, and also prevent the edges of the flattened retina from
overlapping with each other on the surface of the sphere. The method
as developed so far does incorporate elasticity, but the prevention of
overlap has not yet been implemented.

\section{Algorithm}
\label{fold-retina:sec:method}

\subsection{Triangulation of flattened retina}
\label{fold-sphere:sec:triang-flatt-retina}

\todo{Write the actual procedure for triangulation}

An equilateral triangular grid, where each link has length $L$, is
laid over the surface of the flattened retina
(Figure~\ref{fold-sphere:fig:test1}). At present the grid does not
fill the area entirely, but it would be possible to acheive this by
creating shorter links from the edge of the current grid to the edge
of the folded retina. 

The starting point is $n$ fixed points on the boundary
$\vec{P}_1,\dots,\vec{P}_n$. To this is added $N-n$ internal points
$\vec{P}_{n+1},\dots,\vec{P}_N$. The triangulation comprises $M$
triangles and is described by an $M\times3$ matrix $T$, where each row
contains the indicies of the points in the triangle in anticlockwise order.

The connections are defined by $C_{ij}$, a symmetric, binary-valued
matrix that defines if there is a connection between $i$ and $j$ and
$L_{ij}$ are the distances between grid points on the flattened
retina.

\subsection{Projection onto hemisphere}
\label{fold-sphere:sec:proj-onto-hemisph}

It is supposed that this grid is to be projected onto a sphere with a
radius appropriate for the area $A$ of the flattened retina. The
radius is
\begin{equation}
  \label{fold-sphere:eq:1}
  R = \sqrt{\frac{A}{2\pi\sin\phi_0+1}}
\end{equation}
where $\phi_0$ is the latitude at which the rim of the intact retina
would be expected.

The aim now is to infer the latitude $\phi_i$ and longitude
$\lambda_i$ on the sphere to which each grid point $i$ on the
flattened retina is projected.  It is proposed to achieve this aim by
minimising an energy function which depends on three components, an
elastic component $E_\mathrm{E}$, an area-preserving component
$E_\mathrm{A}$ and a dipole component $E_\mathrm{D}:$
\begin{equation}
  \begin{split}
  E(\phi_1,\dots,\phi_N,\lambda_1,\dots,\lambda_N) = & \\
  E_\mathrm{E}(\phi_1,\dots,\phi_N,\lambda_1,\dots,\lambda_N) 
  & + E_\mathrm{A}(\phi_1,\dots,\phi_N,\lambda_1,\dots,\lambda_N) 
  + E_\mathrm{D}(\phi_1,\dots,\phi_n,\lambda_1,\dots,\lambda_n)
  \end{split}
\end{equation}

\subsection{The elastic energy}
\label{fold-sphere:sec:elastic-force}

This component of the energy corresponds to the energy contained in
imageinary springs with the natural lenght of the distances in the
flattened retina, $L_{ij}$:
\begin{equation}
  \label{fold-sphere:eq:5}
  \EE  = \frac{1}{2} \sum_{i=1}^N \sum_{j=1}^N C_{ij} (l_{ij}-L_{ij})^2  
\end{equation}
where $l_{ij}$ is the distance between grid points on the sphere and
is given by the formula for the central angle:
\begin{equation}
  \label{fold-sphere:eq:2}
  l_{ij} = R\cos^{-1}(\cos\phi_i\cos\phi_j\cos(\lambda_i-\lambda_j) + \sin\phi_i\sin\phi_j)
\end{equation}
Minimising this energy function should lead to the distances between
neighbouring points on the sphere being similar to the corresponding
distances on the flattened retina.

In order to minimise the function efficiently, the derivatives with
respect to $\phi_i$ and $\lambda_i$ are found:
\begin{equation}
  \label{fold-sphere:eq:3}
  \begin{split}
    \frac{\partial \EE}{\partial\phi_i} = 
    \sum_j C_{ij} (l_{ij} - L_{ij})R
    \frac{\sin\phi_i\cos\phi_j\cos(\lambda_i-\lambda_j) - \cos\phi_i\sin\phi_j}
    {\sqrt{1-(\sin\phi_i\sin\phi_j +
        \cos\phi_i\cos\phi_j\cos(\lambda_i-\lambda_j))^2}} \\
    \frac{\partial \EE}{\partial\lambda_i} = 
    \sum_j C_{ij} (l_{ij} - L_{ij})R
    \frac{\cos\phi_i\cos\phi_j\sin(\lambda_i-\lambda_j)}
    {\sqrt{1-(\sin\phi_i\sin\phi_j + \cos\phi_i\cos\phi_j\cos(\lambda_i-\lambda_j))^2}}
  \end{split}
\end{equation}
A quasi-Newton-Raphson method is then used in R to achieve this
optimisation, and the resulting grid on the sphere is plotted in 3D
(Figure~\ref{fold-sphere:fig:test2}).

\subsection{The area-preserving energy}
\label{fold-sphere:sec:area-pres-energy}

\subsection{The dipole energy}
\label{fold-sphere:sec:dipole-energy}

The total dipole energy is
\begin{equation}
  \label{fold-sphere:eq:6}
  \ED = \sum_{j=1}^n \sum_{k=1, k\ne j, k\ne j+1}^n \Ed(\p_j,
    \p_{j+1}, \p_k)
\end{equation}
where $\Ed(\p_j,\p_{j+1}, \p_k)$ is the energy due to
the attraction of the line segment between adjacent points $\p_j$ and
$\p_{j+1}$ on the boundary and the point $\p_k$.

The dipole energy is derived by integrating the point energy function of two
points separated by $r$:
\begin{equation}
  \frac{1}{1 + (r/d)^2}
\end{equation}
over the line segment
\begin{equation}
  \label{fold-sphere:eq:7}
  \Ed(\p_1, \p_2, \p_3) = -\int_{s_1}^{s_2}  \frac{1}{1 + (r/d)^2} ds
\end{equation}
where $r = \sqrt{r_0^2 + s^2}$ and
\begin{equation}
  \label{fold-sphere:eq:8}
  r_0 = \frac{1}{R^2}(\p_1 \times \p_2) \cdot \p_3
\end{equation}
and 
\begin{equation}
  s_1 = (\p_1 - \p_3)\cdot
  \frac{\p_2-\p_1}{|\p_2-\p_1|} \quad
  s_2 = (\p_2 - \p_3)\cdot
  \frac{\p_2-\p_1}{|\p_2-\p_1|}
\end{equation}
Solving~(\ref{fold-sphere:eq:7}) gives
\begin{equation}
  \label{fold-sphere:eq:9}
  \Ed(\p_1, \p_2, \p_3) = 
-\frac{d^2}{s_0}\left[\tan^{-1}\frac{s}{s_0}\right]_{s_1}^{s_2}
\end{equation}
where $s_0=\sqrt{d^2 + r_0^2}$

Differentiate:
\begin{equation}
  \label{fold-sphere:eq:10}
  \frac{\partial \ED}{\partial \p_i} = 
  \sum_{k=1, k\ne i, k\ne i+1}^n 
  \frac{\partial \Ed(\p_i, \p_{i+1}, \p_k)}{\partial \p_i}
  + \sum_{k=1, k\ne i-1, k\ne i}^n 
  \frac{\partial\Ed(\p_{i-1}, \p_{i}, \p_k)}{\partial \p_i}
  + \sum_{j=1, j\ne i-1, j\ne i}^n 
  \frac{\partial \Ed(\p_{j}, \p_{j+1}, \p_i)}{\partial \p_i}
\end{equation}

\begin{equation}
  \label{fold-sphere:eq:11}
  \frac{\partial \Ed(\p_i, \p_{i+1}, \p_k)}{\partial \p_i} =
  \frac{\partial \Ed}{\partial r_0} \frac{r_0(\p_i, \p_{i+1}, \p_k)}{\p_i}
\end{equation}

\begin{equation}
  \label{fold-sphere:eq:12}
  \frac{\partial r_0(\p_i, \p_{i+1}, \p_k)}{\partial \p_i} = 
  \frac{1}{R^2} \p_{i+1} \times \p_k \quad
  \frac{\partial r_0(\p_{i-1}, \p_{i}, \p_k)}{\partial \p_i} = 
  \frac{1}{R^2} \p_{k} \times \p_{i-1} \quad
  \frac{\partial r_0(\p_{j}, \p_{j+1}, \p_i)}{\partial \p_i} = 
  \frac{1}{R^2} \p_{j} \times \p_{j+1} 
\end{equation}

\begin{equation}
  \frac{\partial s_1}{\partial \p_1} = 
  \frac{\p_2 - \p_1}{|\p_2 - \p_1|} \quad
  \frac{\partial s_2}{\partial \p_1} = \vec{0}
\end{equation}

\begin{equation}
  \frac{\partial s_1}{\partial \p_2} = \vec{0} \quad
  \frac{\partial s_2}{\partial \p_2} =   \frac{\p_2 - \p_1}{|\p_2 - \p_1|} 
\end{equation}

\begin{equation}
  \frac{\partial s_1}{\partial \p_3} = 
  -\frac{\p_2 - \p_1}{|\p_2 - \p_1|} \quad
  \frac{\partial s_2}{\partial \p_3} = 
  -\frac{\p_2 - \p_1}{|\p_2 - \p_1|} \quad
\end{equation}




\begin{equation}
  \label{fold-sphere:eq:13}
\end{equation}

% \section{Application to retina}
% \label{fold-retina:sec:application-retina}

% Figure~\ref{fold-sphere:fig:test1}. 
% Figure~\ref{fold-sphere:fig:test2}. 

\begin{figure}
  \centering
  \includegraphics[width=0.6\linewidth]{flattened-grid}
  \includegraphics[width=0.6\linewidth]{test1-sphere}
  \caption{The basic algorithm applied to Dan's data. Top: the grid on
    the flattened retina. Bottom: the mapping of the grid onto the
    sphere. $\phi_0= 60^\circ$.}
  \label{fold-sphere:fig:test1}
\end{figure}

\begin{figure}
  \centering
  \includegraphics[width=0.6\linewidth]{test2-sphere}
  \caption{The same retina as in Figure~\ref{fold-sphere:fig:test1},
    but mapped onto a partial sphere with $\phi_0= 50^\circ$.}
  \label{fold-sphere:fig:test2}
\end{figure}

\begin{equation}
  \ED = \sum_{j=1}^n \sum_{k=1}^n \ED^{jk}
\end{equation}

\begin{equation}
  \begin{split}
  \frac{\partial \ED}{\partial \p_i} = 
  \sum_{k=1}^n & 
  \frac{\partial \ED^{ik}}{\partial r_0^{ik}}
  \frac{\partial r_0^{ik}}{\partial \p_i} +
  \frac{\partial \ED^{ik}}{\partial s_1^{ik}}
  \frac{\partial s_1^{ik}}{\partial \p_i} +
  \frac{\partial \ED^{ik}}{\partial s_2^{ik}}
  \frac{\partial s_2^{ik}}{\partial \p_i} \\
  &   
  + \frac{\partial \ED^{i-1k}}{\partial r_0^{ik}}
  \frac{\partial r_0^{i-1k}}{\partial \p_i} +
  \frac{\partial \ED^{i-1k}}{\partial s_1^{i-1k}}
  \frac{\partial s_1^{i-1k}}{\partial \p_i} +
  \frac{\partial \ED^{ik}}{\partial s_2^{ik}}
  \frac{\partial s_2^{i-1k}}{\partial \p_i} \\
  & 
  + \frac{\partial \ED^{ki}}{\partial r_0^{ki}}
  \frac{\partial r_0^{ki}}{\partial \p_i} +
  \frac{\partial \ED^{ki}}{\partial s_1^{ki}}
  \frac{\partial s_1^{ki}}{\partial \p_i} +
  \frac{\partial \ED^{ki}}{\partial s_2^{ki}}
  \frac{\partial s_2^{ki}}{\partial \p_i} \\
  \end{split}
\end{equation}



\section{Discussion}
\label{fold-retina:sec:discussion}

Figures~\ref{fold-sphere:fig:test1} and~\ref{fold-sphere:fig:test2}
show that the algorithm does project the flattened retina onto a
sphere. The edges of the retina at the outer ends of tears are closer
together on the sphere than they are in the flattened coordinate
system. However, towards the base of the tears, there is considerable
overlap between neighbouring ``flaps'' of the flattened retina. The
algorithm as it stands is therefore not sufficient to allocate every
point on the flattened retina a coordinate in the presumed sphere of
the intact retina.

The reason for these overlaps is that there is no component of the
energy function that prevents them occurring. I envisage that it would
be possible to put in a component of the energy function that provides
for short range repulsion between vertices on the edge of the
flattened retina and edges of the flattened retina. The energy
function might have longer range attraction. One possible
parametrisation would be the Lennard-Jones potential which is used to
model short range repulsion and longer range attraction between
molecules:
\begin{equation}
  \label{fold-sphere:eq:4}
  E(r) = 4\epsilon\left(\left(\frac{\sigma}{r}\right)^{12}-
    \left(\frac{\sigma}{r}\right)^{6}\right)
\end{equation}
where $r$ is the distance between molecules, $\epsilon$ defines the
depth of the potential well and $\sigma$ is the distance at which the
potential is zero.

A useful preliminary step before implementing this would be to improve
the algorithm that creates the triangular mesh over the flattened
retina, so that it completely fills the space by making links to the
edge of the retina and by laying links along the edge of the retina.

Were this approach to work, additional refinements might be possible,
such as adding a component of the energy function that draws together
points on either side of a rip which are correspond to each other with
high probability. Another possibility would be to allow the radius of
the sphere $R$ to vary within realistic bounds, to optimise the fit
further.

\end{document}

% LocalWords:  ij BP Raphson PDF
%%% Local Variables: 
%%% TeX-PDF-mode: t
%%% End: 
